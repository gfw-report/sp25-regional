% This template file contains packages and commands that are useful and generic
% across papers.
%
% When adding new commands, you can most likely do that within paper.tex, but
% when adding new packages, you might need to update that here to ensure the
% packages are included in the proper order (a common source of compilation
% errors with LaTeX).

%%%%%%%%%%%%%%%%%%%%%%%%%%%%%%%%%%%%%%%%%%%%%%%%%%%%%%%%%%%%%%%%%%%%%%%%%%%%
% Packages {{{

\usepackage[T1]{fontenc}

\usepackage{color}

% Insert images
\usepackage{graphicx}

% Subfigures.
\usepackage[labelformat=simple]{subcaption}
\usepackage{xspace}
\usepackage{multirow}
\usepackage{makecell} % Not sure if we need this
\usepackage{enumitem} % We do need this for IP Order appendix
%\usepackage[ruled,vlined]{algorithm2e}

\usepackage{ulem}
\normalem

\usepackage{listings}
\lstset{breaklines=true}

\usepackage[export]{adjustbox}
\usepackage{multirow}
\usepackage{booktabs}

\usepackage{algpseudocode}
\usepackage{algorithm}

\usepackage{bigdelim}
\usepackage{changepage}
\usepackage{float}

%%%%%%%%%%%%%%%%%%%%%%%%%%%%%%%%%%%%%%%%%%%%%%%%%%%%%%%%%%%%%%%%%%%%%%%%%%%%
% Only enable this for footnote-style affiliations
%\usepackage{authblk}

% }}}
%%%%%%%%%%%%%%%%%%%%%%%%%%%%%%%%%%%%%%%%%%%%%%%%%%%%%%%%%%%%%%%%%%%%%%%%%%%%
% Commands {{{

\newcommand{\etal}{et~al.\xspace}

\renewcommand\thesubfigure{(\alph{subfigure}) }



% }}}
%%%%%%%%%%%%%%%%%%%%%%%%%%%%%%%%%%%%%%%%%%%%%%%%%%%%%%%%%%%%%%%%%%%%%%%%%%%%
% Environments {{{

\newenvironment{strategy}[1][htb]
  {\renewcommand{\algorithmcfname}{Strategy}
   \begin{algorithm}[#1]%
  }{
   \end{algorithm}
  }

\newenvironment{tightlist}{
	\begin{list}{$\bullet$} {
		\setlength{\itemsep}{0.00cm}
		\setlength{\parskip}{-0.1cm}
	}}{\end{list}}

\newenvironment{widelist}{
	\begin{list}{$\bullet$} {
		\setlength{\leftmargin}{.60cm}
		\setlength{\itemsep}{.00cm}
	}}{\end{list}}

% }}}
%%%%%%%%%%%%%%%%%%%%%%%%%%%%%%%%%%%%%%%%%%%%%%%%%%%%%%%%%%%%%%%%%%%%%%%%%%%%
% Comment Commands {{{

\ifdefined\isFinalized
\newcommand{\NewCommentType}[3]{}
\else
\newcommand{\NewCommentType}[3]{\expandafter\newcommand\csname #1\endcsname[1]{{\color{#2}{#3: ##1}} }}
\fi

% }}}
%%%%%%%%%%%%%%%%%%%%%%%%%%%%%%%%%%%%%%%%%%%%%%%%%%%%%%%%%%%%%%%%%%%%%%%%%%%%
% Line-break nudging {{{

\clubpenalty=10000
\widowpenalty = 10000

\hyphenation{de-a-non-y-mi-za-tion}
\hyphenation{none-the-less}
\hyphenation{libev}
\hyphenation{Shadow-socks}
\hyphenation{Outline-VPN}
\hyphenation{Web-RTC}
\hyphenation{Geo-Lite}
\hyphenation{Ali-ba-ba}
\hyphenation{time-stamp}
\hyphenation{time-stamps}

% }}}
%%%%%%%%%%%%%%%%%%%%%%%%%%%%%%%%%%%%%%%%%%%%%%%%%%%%%%%%%%%%%%%%%%%%%%%%%%%%
% Prettier fonts {{{

\usepackage[override]{cmtt} % make tt font tighter / less ugly

% To render CU tap figure
\usepackage[utf8]{inputenc}
\usepackage{pgfplots}
\DeclareUnicodeCharacter{2212}{−}
\usepgfplotslibrary{groupplots,dateplot}
\usetikzlibrary{patterns,shapes.arrows}
\pgfplotsset{compat=newest}


% }}}
%%%%%%%%%%%%%%%%%%%%%%%%%%%%%%%%%%%%%%%%%%%%%%%%%%%%%%%%%%%%%%%%%%%%%%%%%%%%

%% begin self-added packages

%% % For citations having Chinese characters in the title
\usepackage{CJKutf8}

%%%%%%%%%%%%%%%%%%%%%%%%%%%%%%%%%%%%%%%%%%%%%%%%%%%%%%%%%%%%%%%%%%%%%%%%%%%%

% For tables
\usepackage{multirow}
\usepackage{booktabs}
\usepackage{algpseudocode}
\usepackage{algorithm}
\usepackage{caption}
\usepackage[export]{adjustbox}
\usepackage{siunitx}

%%%%%%%%%%%%%%%%%%%%%%%%%%%%%%%%%%%%%%%%%%%%%%%%%%%%%%%%%%%%%%%%%%%%%%%%%%%%

% Don't use monospace for URLs.
%\urlstyle{same}

%%%%%%%%%%%%%%%%%%%%%%%%%%%%%%%%%%%%%%%%%%%%%%%%%%%%%%%%%%%%%%%%%%%%%%%%%%%%

% To balance references
\usepackage{balance}

%%%%%%%%%%%%%%%%%%%%%%%%%%%%%%%%%%%%%%%%%%%%%%%%%%%%%%%%%%%%%%%%%%%%%%%%%%%%
% Autoref {{{

% Labels for \autoref.
% https://tex.stackexchange.com/a/313508
\def\sectionautorefname{Section}
\def\subsectionautorefname{Section}
\def\algorithmautorefname{Algorithm}

%%%%%%%%%%%%%%%%%%%%%%%%%%%%%%%%%%%%%%%%%%%%%%%%%%%%%%%%%%%%%%%%%%%%%%%%%%%%

%% Parhead
\newcommand{\parhead}[1]{\medskip\Parhead{#1}}
% \Parhead is like \parhead except that it doesn't add preceding vertical space.
\newcommand{\Parhead}[1]{\noindent\textbf{#1}\hskip 0.5em\relax}

%%%%%%%%%%%%%%%%%%%%%%%%%%%%%%%%%%%%%%%%%%%%%%%%%%%%%%%%%%%%%%%%%%%%%%%%%%%%
%% Comments/Notes
%% \newcommand{\todo}[1]{{\color{red}{TODO TK: #1}} }
%% \newcommand{\dml}[1]{{\color{blue}{[[Dave: #1]]}} }
 \newcommand{\note}[1]{{\color{blue}{[[#1]]}} }

%%%%%%%%%%%%%%%%%%%%%%%%%%%%%%%%%%%%%%%%%%%%%%%%%%%%%%%%%%%%%%%%%%%%%%%%%%%%
% Anonymization for CU
\newcommand{\cublong}{University of Colorado Boulder}
%\newcommand{\cublong}{University }
\newcommand{\cub}{CU Boulder}
%\newcommand{\cub}{University }

\newcommand{\Q}[1]{{\color{red}{Question: #1}} }

%%%%%%%%%%%%%%%%%%%%%%%%%%%%%%%%%%%%%%%%%%%%%%%%%%%%%%%%%%%%%%%%%%%%%%%%%%%%
% Disable metadata for reproducible PDF.
% https://tex.stackexchange.com/a/313605
\ifpdf
\pdfinfoomitdate=1
\pdftrailerid{}
\pdfsuppressptexinfo=-1
%\hypersetup{pdfcreator={},pdfproducer={}}
\fi
%%%%%%%%%%%%%%%%%%%%%%%%%%%%%%%%%%%%%%%%%%%%%%%%%%%%%%%%%%%%%%%%%%%%%%%%%%%%
% Suppress page numbers for camera-ready version.
% \pagenumbering{gobble}

%%%%%%%%%%%%%%%%%%%%%%%%%%%%%%%%%%%%%%%%%%%%%%%%%%%%%%%%%%%%%%%%%%%%%%%%%%%%

% Use an unbreakable space before the section number in citations like
% \cite[\S 1.2]{Ref}, rather than the default comma plus breakable space.
\renewcommand{\citemid}{~}

%%%%%%%%%%%%%%%%%%%%%%%%%%%%%%%%%%%%%%%%%%%%%%%%%%%%%%%%%%%%%%%%%%%%%%%%%%%%
% Paper-specific commands {{{

% System name (if applicable)
\newcommand{\name}{$\mathsf{SystemName}$\xspace} % middle of sentence
\newcommand{\Name}{$\mathsf{SystemName}$\xspace} % start of sentence

% Per-author comment commands (they don't appear if you "make finalized")
% 	Colors: black, blue, brown, cyan, darkgray, gray, green, lightgray, lime
% 	magenta, olive, orange, pink, purple, red, teal, violet, white,	yellow
\NewCommentType{todo}{red}{TODO}
\NewCommentType{dml}{violet}{dml}

% }}}
%%%%%%%%%%%%%%%%%%%%%%%%%%%%%%%%%%%%%%%%%%%%%%%%%%%%%%%%%%%%%%%%%%%%%%%%%%%%

% Helpful self-defined command for IEEE author block alignment:
% https://tex.stackexchange.com/questions/458204/ieeetran-document-class-how-to-align-five-authors-properly/458208#458208

\makeatletter
\newcommand{\linebreakand}{%
  \end{@IEEEauthorhalign}
  \hfill\mbox{}\par
  \mbox{}\hfill\begin{@IEEEauthorhalign}
}
\makeatother

% For equal contribution footnote

\makeatletter
\newcommand{\footnotestar}[1]{%
  \begingroup%
  \renewcommand{\thefootnote}{*}%
  \renewcommand{\@makefnmark}{\hbox{\textsuperscript{*}}}%
  \footnotetext{#1}%
  \endgroup%
}
\makeatother


%%%%%%%%%%%%%%%%%%%%%%%%%%%%%%%%%%%%%%%%%%%%%%%%%%%%%%%%%%%%%%%%%%%%%%%%%%%%

% Environment for drawing a sequence of byte-sized boxes around text.
% Use \x for a one-byte box, \xx for a two-byte box, \xxxx for a four-byte box,
% and \xxxxxxxx for an eight-byte box.
% Example:
% \begin{bytebox}
% \x{03}\x{c}\x{o}\x{m}
% \xx{c00c}
% \end{bytebox}

\newlength{\byteboxdim}
\newlength{\byteboxwidth}
% \byteboxpre is empty on entering the bytebox environment. It is overridden in
% \byteboxpost to permit line breaks between cells. No line break is permitted
% before the first cell, in order to keep e.g. opening parentheses attached if adjacent.
\newcommand{\byteboxpre}{}
\newcommand{\byteboxpost}{%
% permit line break before the next cell, if any
\renewcommand{\byteboxpre}{\hspace{0pt}}%
% space backwards for the next cell
\kern-\fboxrule\relax%
\ignorespaces\nopagebreak%
}
\newenvironment{bytebox}{%
\renewcommand{\byteboxpre}{}%
% Size byte cells relative to the size of \strutbox in the current font size.
\setlength{\byteboxdim}{\dimexpr\dp\strutbox + \ht\strutbox}%
\setlength{\byteboxwidth}{\byteboxdim}%
\setlength{\fboxsep}{0pt}%
\ignorespaces%
}{%
\nolinebreak\kern\fboxrule\relax%
}
\newcommand{\x}[2][white]{%
\byteboxpre%
\fcolorbox{frame}{#1}{\makebox[\byteboxwidth]{\rule[-0.2\byteboxdim]{0pt}{0.9\byteboxdim}{\texttt{#2}}}}%
\byteboxpost%
}
\newcommand{\xx}[2][white]{%
\byteboxpre%
\fcolorbox{frame}{#1}{\makebox[\dimexpr 2\byteboxwidth+\fboxrule]{\rule[-0.2\byteboxdim]{0pt}{0.9\byteboxdim}{\texttt{#2}}}}%
% draw tick marks within the box
{%
\color{frame}%
\llap{\rule[-0.2\byteboxdim]{\fboxrule}{0.1\byteboxdim}\hspace{\dimexpr \byteboxwidth+\fboxrule}}%
\llap{\rule[+0.6\byteboxdim]{\fboxrule}{0.1\byteboxdim}\hspace{\dimexpr \byteboxwidth+\fboxrule}}%
}%
\byteboxpost%
}
\newcommand{\xxxx}[2][white]{%
\byteboxpre%
\fcolorbox{frame}{#1}{\makebox[\dimexpr 4\byteboxwidth+3\fboxrule]{\rule[-0.2\byteboxdim]{0pt}{0.9\byteboxdim}{\texttt{#2}}}}%
% draw tick marks within the box
{%
\color{frame}%
\llap{\rule[-0.2\byteboxdim]{\fboxrule}{0.1\byteboxdim}\hspace{\dimexpr 1\byteboxwidth+1\fboxrule}}%
\llap{\rule[-0.2\byteboxdim]{\fboxrule}{0.1\byteboxdim}\hspace{\dimexpr 2\byteboxwidth+2\fboxrule}}%
\llap{\rule[-0.2\byteboxdim]{\fboxrule}{0.1\byteboxdim}\hspace{\dimexpr 3\byteboxwidth+3\fboxrule}}%
\llap{\rule[+0.6\byteboxdim]{\fboxrule}{0.1\byteboxdim}\hspace{\dimexpr 1\byteboxwidth+1\fboxrule}}%
\llap{\rule[+0.6\byteboxdim]{\fboxrule}{0.1\byteboxdim}\hspace{\dimexpr 2\byteboxwidth+2\fboxrule}}%
\llap{\rule[+0.6\byteboxdim]{\fboxrule}{0.1\byteboxdim}\hspace{\dimexpr 3\byteboxwidth+3\fboxrule}}%
}%
\byteboxpost%
}
\newcommand{\xxxxxxxx}[2][white]{%
\byteboxpre%
\fcolorbox{frame}{#1}{\makebox[\dimexpr 8\byteboxwidth+7\fboxrule]{\rule[-0.2\byteboxdim]{0pt}{0.9\byteboxdim}{\texttt{#2}}}}%
% draw tick marks within the box
{%
\color{frame}%
\llap{\rule[-0.2\byteboxdim]{\fboxrule}{0.1\byteboxdim}\hspace{\dimexpr 1\byteboxwidth+1\fboxrule}}%
\llap{\rule[-0.2\byteboxdim]{\fboxrule}{0.1\byteboxdim}\hspace{\dimexpr 2\byteboxwidth+2\fboxrule}}%
\llap{\rule[-0.2\byteboxdim]{\fboxrule}{0.1\byteboxdim}\hspace{\dimexpr 3\byteboxwidth+3\fboxrule}}%
\llap{\rule[-0.2\byteboxdim]{\fboxrule}{0.1\byteboxdim}\hspace{\dimexpr 4\byteboxwidth+4\fboxrule}}%
\llap{\rule[-0.2\byteboxdim]{\fboxrule}{0.1\byteboxdim}\hspace{\dimexpr 5\byteboxwidth+5\fboxrule}}%
\llap{\rule[-0.2\byteboxdim]{\fboxrule}{0.1\byteboxdim}\hspace{\dimexpr 6\byteboxwidth+6\fboxrule}}%
\llap{\rule[-0.2\byteboxdim]{\fboxrule}{0.1\byteboxdim}\hspace{\dimexpr 7\byteboxwidth+7\fboxrule}}%
\llap{\rule[+0.6\byteboxdim]{\fboxrule}{0.1\byteboxdim}\hspace{\dimexpr 1\byteboxwidth+1\fboxrule}}%
\llap{\rule[+0.6\byteboxdim]{\fboxrule}{0.1\byteboxdim}\hspace{\dimexpr 2\byteboxwidth+2\fboxrule}}%
\llap{\rule[+0.6\byteboxdim]{\fboxrule}{0.1\byteboxdim}\hspace{\dimexpr 3\byteboxwidth+3\fboxrule}}%
\llap{\rule[+0.6\byteboxdim]{\fboxrule}{0.1\byteboxdim}\hspace{\dimexpr 4\byteboxwidth+4\fboxrule}}%
\llap{\rule[+0.6\byteboxdim]{\fboxrule}{0.1\byteboxdim}\hspace{\dimexpr 5\byteboxwidth+5\fboxrule}}%
\llap{\rule[+0.6\byteboxdim]{\fboxrule}{0.1\byteboxdim}\hspace{\dimexpr 6\byteboxwidth+6\fboxrule}}%
\llap{\rule[+0.6\byteboxdim]{\fboxrule}{0.1\byteboxdim}\hspace{\dimexpr 7\byteboxwidth+7\fboxrule}}%
}%
\byteboxpost%
}

% Colors available to use as the optional argument to byte \x, \xx, \xxxx.
% Hint: use the online oklch color picker to adjust lightness and chroma and convert back to sRGB.
\definecolor{frame}{gray}{0.40}       % box frame
\definecolor{h} {RGB}{242,242,242} % https://oklch.com/#96,0.000,360,100 header
\definecolor{hc}{RGB}{242,242,242} % https://oklch.com/#96,0.000,360,100 header count fields: QDCOUNT, ANCOUNT, NSCOUNT, ARCOUNT
\definecolor{lp}{RGB}{111,246,253} % https://oklch.com/#90,0.118,200,100 label length prefix
\definecolor{l} {RGB}{204,253,255} % https://oklch.com/#96,0.118,200,100 label byte
\definecolor{q} {RGB}{245,243,224} % https://oklch.com/#96,0.024,100,100 QTYPE, QCLASS
\definecolor{in}{RGB}{252,229,229} % https://oklch.com/#94,0.026,17,100  injected answer resource record
\definecolor{dg}{RGB}{220,254,206} % https://oklch.com/#96,0.072,136,100 "digest" bytes that are the first 4 leaked bytes in some cases
\definecolor{b} {RGB}{247,243,211} % https://oklch.com/#96,0.042,102,100 Wallbleed leaked bytes of memory

%%%%%%%%%%%%%%%%%%%%%%%%%%%%%%%%%%%%%%%%%%%%%%%%%%%%%%%%%%%%%%%%%%%%%%%%%%%%

%% To use \cmark and \xmark
\usepackage{pifont}
\newcommand{\cmark}{\ding{51}}%
\newcommand{\xmark}{\ding{55}}%

%% end self-added packages
